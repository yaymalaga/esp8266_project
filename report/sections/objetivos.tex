\section{Objetivos}\label{sec:objetivos}
Los objetivos han sido extraídos del enunciado del proyecto. Como se ha explicado en \ref{sec:introduccion}, el objetivo principal es facilitar la obtención de datos de sensores instalados en un huerto urbano gestionado por el departamento de Biología vegetal de la Universidad de Málaga. Para ello se utilizará el uso de Internet de las cosas.

\subsection{Objetivos mínimos}

\subsection{Objetivos adicionales}

\subsection{Método}

Debido a la modularidad en el diseño del sistema, optamos por dividir los objetivos entre los integrantes del grupo. Cada uno de ellos ha asumido algunas de las tareas obligatorias primero. Una vez completadas estas, se asumieron tareas opcionales directamente relacionadas o por preferencia del integrante. Las tareas de coordinación y dirección, así como los responsables individuales de cada documento (código Arduino, Node-RED, memoria...). Nos encontramos por lo tanto frente a flujo que combina las ventajas del trabajo concurrente con el control de integración en cada documento por parte de los responsables. Para cada tarea se ha seguido un método en cascada, por el cual se iban implementando funcionalidades y validando antes de pasar a la siguiente tarea relacionada.

Para todo el código se ha utilizado el control de versiones de git en GitHub y el código se encuentra disponible. Esto ha facilitado los métodos anteriormente mencionados evitando conflictos de código y permitiendo un desarrollo orgánico con posibilidad de recuperación.


