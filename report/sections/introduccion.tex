\section{Introducción}\label{sec:introduccion}
En la actualidad, el internet de las cosas (en inglés, \textit{Internet of things}, abreviado \textit{IoT}) ha ganado un importante lugar en la sociedad. Los avances en tecnología han ofrecido grandes mejoras en procesamiento de datos y conectividad; dando lugar a una era en la que la información y su uso se han elevado a un nuevo plano \parencite{cisco:idc}.

Podemos considerar el IoT como una evolución del ya establecido Internet. La interconexión de objetos cotidianos como nodos distribuidos, se pueden adquirir una gran cantidad de medidas. Posteriormente, estas medidas se podrán procesar y controlar distintos dispositivos en base a ellas. El hecho de que esta comunicación se realice entre cosas en lugar de personas, supone un uso obligado de la automatización para explotar toda su eficacia. Por ello, en la asignatura de Informática Industrial, impartida en el grado de Electrónica, Robótica y Mecatrónica de la Universidad de Málaga, se considera que su conocimiento y puesta en práctica es necesaria para la correcta formación de cualquier ingeniero en esta disciplina.

La lista de usos del IoT es interminable. Este conjunto de tecnologías permite mejorar la calidad de vida de las personas. Desde el día a día con aplicaciones en domótica o ayudas en tareas cruciales como medicina o seguridad. Otras áreas de aplicación pueden ser la industria, el transporte o la agricultura. Fundamentalmente, estas nuevas ciencias redefinirán el uso de la tecnología como lo conocemos.

En nuestro caso particular, mediante este proyecto se plantea mejorar el proceso de recogida y procesado de datos de un huerto urbano gestionado por el departamento de Biología vegetal de la universidad de Málaga. Este proceso se realizaba hasta ahora por los investigadores y alumnos de asignaturas relacionadas. Para ello, se propuso el uso de Internet de las cosas. El sistema está formado por una red de "nodos sensores" que recogen datos útiles para los estudios del huerto y una estación que recibe, procesa y muestra la información de una forma amigable para los usuarios \parencite{uma:enunciado}. 