\documentclass[paper=a4, fontsize=11pt]{scrartcl}

%%%%%%%%%%%%%%%%%%%%%%%%%%%%%%%%%%%%%%%%%%%%%%%%%%%%%%%%%%%%%%%%%%%%
%%-----------------------PAGE SETTINGS----------------------------%%
%%%%%%%%%%%%%%%%%%%%%%%%%%%%%%%%%%%%%%%%%%%%%%%%%%%%%%%%%%%%%%%%%%%%
\usepackage[utf8]{inputenc}
\usepackage[T1]{fontenc} %To support Spanish characters
\usepackage[spanish]{babel}
\def\spanishoperators{}
\usepackage{csquotes}
\usepackage[a4paper, margin= 1in,includefoot]{geometry} % Page dimensions
%%% Custom sectioning
\usepackage{sectsty}
\allsectionsfont{\centering}
%%%%%%%%%%%%%%%%%%%%%%%%%%%%%%%%%%%%%%%%%%%%%%%%%%%%%%%%%%%%%%%%%%%%
%%--------------------------PREAMBLE------------------------------%%
%%%%%%%%%%%%%%%%%%%%%%%%%%%%%%%%%%%%%%%%%%%%%%%%%%%%%%%%%%%%%%%%%%%%
% Support the \includegraphics command and options
\usepackage{graphicx} 
%---------------------------------------------------------------------------------------------%
% Package for managing IS units
\usepackage{siunitx}
%---------------------------------------------------------------------------------------------%
% Package for bibliography
\usepackage{biblatex}
\addbibresource{references.bib}
%---------------------------------------------------------------------------------------------%

% Microtype uses different techniques to improve readability and presentation. It is and advanced plugin. Some documentation can be found: http://www.khirevich.com/latex/microtype/

\usepackage[activate={true,nocompatibility},final,tracking=true,kerning=true,spacing=true,factor=1100,stretch=10,shrink=10]{microtype}
% activate={true,nocompatibility} - activate protrusion and expansion
% final - enable microtype; use "draft" to disable
% tracking=true, kerning=true, spacing=true - activate these techniques
% factor=1100 - add 10% to the protrusion amount (default is 1000)
% stretch=10, shrink=10 - reduce stretchability/shrinkability (default is 20/20)

\SetExtraKerning[unit=space]
    {encoding={*}, family={bch}, series={*}, size={footnotesize,small,normalsize}}
    {\textendash={400,400}, % en-dash, add more space around it
     "28={ ,150}, % left bracket, add space from right
     "29={150, }, % right bracket, add space from left
     \textquotedblleft={ ,150}, % left quotation mark, space from right
     \textquotedblright={150, }} % right quotation mark, space from left

% Additional packages for customization
% TODO
\usepackage{caption}
\usepackage{subcaption}
%\usepackage{lmodern}
%\usepackage{times}
%\usepackage[expert]{mathdesign}
% \usepackage[parfill]{parskip} % Activate to begin paragraphs with an empty line rather than an indent
%\usepackage{booktabs} % for much better looking tables
%\usepackage{array} % for better arrays (eg matrices) in maths
%\usepackage{paralist} % very flexible & customised lists (eg. enumerate/itemize, etc.)
%\usepackage{verbatim} % adds environment for commenting out blocks of text & for better verbatim
%\usepackage{subfig} % make it possible to include more than one captioned figure/table in a single float
%\usepackage{lastpage}

%%%%%%%%%%%%%%%%%%%%%%%%%%%%%%%%%%%%%%%%%%%%%%%%%%%%%%%%%%%%%%%%%%%%
%%-----------------------DOCUMENT BEGIN---------------------------%%
%%%%%%%%%%%%%%%%%%%%%%%%%%%%%%%%%%%%%%%%%%%%%%%%%%%%%%%%%%%%%%%%%%%%

%---------------------------------------------------------------------------------------------%

% Title

\title{
    {\huge{\bfseries Huerto inteligente}}\\
    {\large{Informática Industrial}}\\
    }

\author{
    David Morilla Cabello
    \and
    Yeray García Concejero
    \and
   Ana Piñeiro Gómez
    \and
    Miguel Antonio Salazar del Río 
    }

\date{Enero 2020}

%---------------------------------------------------------------------------------------------%

% Header and footer

\usepackage{fancyhdr} % This should be set AFTER setting up the page geometry
\pagestyle{fancy} % options: empty , plain , fancy

%Header
\fancyhead{}
\renewcommand{\headrulewidth}{0.4pt}
\fancyhead[R]{\small \textsf{ESCUELA DE INGENIERÍAS INDUSTRIALES\\ Informática Industrial\\Huerto Inteligente\\}}
\fancyhead[L]{\includegraphics[scale=0.5]{./figuras/logoUMA.png}\\}
\setlength{\headheight}{62pt} % This is specific for the logo

%Footer

\fancyfoot{}
\renewcommand{\footrulewidth}{0.4pt}
\fancyfoot[R]{\thepage}
\fancyfoot[C]{Sección \thesection}

%%%%%%%%%%%%%%%%%%%%%%%%%%%%%%%%%%%%%%%%%%%%%%%%%%%%%%%%%%%%%%%%%%%%
%%-----------------------DOCUMENT BEGIN---------------------------%%
%%%%%%%%%%%%%%%%%%%%%%%%%%%%%%%%%%%%%%%%%%%%%%%%%%%%%%%%%%%%%%%%%%%%
\begin{document}

\begin{titlepage}
\begin{center}
 {\huge\bfseries Proyecto: IoT aplicado al huerto urbano}
 % ----------------------------------------------------------------
 \vspace{2.5cm}
    
 {\large\bfseries  Yeray García Concejero}\\[5pt]
 \texttt{@gmail.com}\\[14pt]
 
  {\large\bfseries David Morilla Cabello}\\[5pt]
 \texttt{davidmorillacabello@gmail.com}\\[14pt]
 
  {\large\bfseries Ana Piñeiro Gómez}\\[5pt]
 \texttt{@gmail.com}\\[14pt]
 
  {\large\bfseries Miguel Antonio Salazar del Río}\\[5pt]
 \texttt{@gmail.com}\\[14pt]
 
  % ----------------------------------------------------------------
 \vspace{2cm}
\emph{{Universidad de Málaga}}\\[2cm]
{Memoria para la asignatura de} \\[2cm]
\textsc{\Large{{Informática Industrial}}} \\[5pt]

% {By}\\[5pt] {\Large \sc {Me}}
 \vfill
 % ----------------------------------------------------------------
\includegraphics[scale=0.75]{figuras/logoUMA.png}\\[5pt]
 \vfill
{Enero 2020}
\end{center}
\end{titlepage}
\pagestyle{plain}
\tableofcontents
\newpage
\pagestyle{fancy}

\section{Introducción}\label{sec:introduccion}
En la actualidad, el internet de las cosas (en inglés, \textit{Internet of things}, abreviado \textit{IoT}) ha ganado un importante lugar en la sociedad. Los avances en tecnología han ofrecido grandes mejoras en procesamiento de datos y conectividad; dando lugar a una era en la que la información y su uso se han elevado a un nuevo plano \parencite{cisco:idc}.

Podemos considerar el IoT como una evolución del ya establecido Internet. La interconexión de objetos cotidianos como nodos distribuidos, se pueden adquirir una gran cantidad de medidas. Posteriormente, estas medidas se podrán procesar y controlar distintos dispositivos en base a ellas. El hecho de que esta comunicación se realice entre cosas en lugar de personas, supone un uso obligado de la automatización para explotar toda su eficacia. Por ello, en la asignatura de Informática Industrial, impartida en el grado de Electrónica, Robótica y Mecatrónica de la Universidad de Málaga, se considera que su conocimiento y puesta en práctica es necesaria para la correcta formación de cualquier ingeniero en esta disciplina.

La lista de usos del IoT es interminable. Este conjunto de tecnologías permite mejorar la calidad de vida de las personas. Desde el día a día con aplicaciones en domótica o ayudas en tareas cruciales como medicina o seguridad. Otras áreas de aplicación pueden ser la industria, el transporte o la agricultura. Fundamentalmente, estas nuevas ciencias redefinirán el uso de la tecnología como lo conocemos.

En nuestro caso particular, mediante este proyecto se plantea mejorar el proceso de recogida y procesado de datos de un huerto urbano gestionado por el departamento de Biología vegetal de la universidad de Málaga. Este proceso se realizaba hasta ahora por los investigadores y alumnos de asignaturas relacionadas. Para ello, se propuso el uso de Internet de las cosas. El sistema está formado por una red de "nodos sensores" que recogen datos útiles para los estudios del huerto y una estación que recibe, procesa y muestra la información de una forma amigable para los usuarios \parencite{uma:enunciado}. 
\section{Objetivos}\label{sec:objetivos}
Los objetivos han sido extraídos del enunciado del proyecto. Como se ha explicado en \ref{sec:introduccion}, el objetivo principal es facilitar la obtención de datos de sensores instalados en un huerto urbano gestionado por el departamento de Biología vegetal de la Universidad de Málaga. Para ello se utilizará el uso de Internet de las cosas.

\subsection{Objetivos mínimos}

\subsection{Objetivos adicionales}

\subsection{Método}

Debido a la modularidad en el diseño del sistema, optamos por dividir los objetivos entre los integrantes del grupo. Cada uno de ellos ha asumido algunas de las tareas obligatorias primero. Una vez completadas estas, se asumieron tareas opcionales directamente relacionadas o por preferencia del integrante. Las tareas de coordinación y dirección, así como los responsables individuales de cada documento (código Arduino, Node-RED, memoria...). Nos encontramos por lo tanto frente a flujo que combina las ventajas del trabajo concurrente con el control de integración en cada documento por parte de los responsables. Para cada tarea se ha seguido un método en cascada, por el cual se iban implementando funcionalidades y validando antes de pasar a la siguiente tarea relacionada.

Para todo el código se ha utilizado el control de versiones de git en GitHub y el código se encuentra disponible. Esto ha facilitado los métodos anteriormente mencionados evitando conflictos de código y permitiendo un desarrollo orgánico con posibilidad de recuperación.



\section{Diseño Hardware}\label{sec:dishard}

\subsection{ESP8266}

Uso del ESP8266. REF al manual. Explicación de sus módulos por encima.

\subsection{DHT11}

Uso del DHT11. Ref al manual. Explicación breve de su func. Añadir imagen.

\subsection{Potenciómetro}

Explicación de su uso como sensor de luminosidad. Explicación breve de su func. Añadir im.

\subsection{Sensor Hum i2c}

Explicación de su uso. Ref al manual. Explicación breve de su func. Añadir im.

\subsection{Reloj tiempo real}

Explicación de su uso. Ref al manual. Explicación breve de su func. Añadir im.

\subsection{Pantalla}

Explicación de su uso. Ref al manual. Explicación breve de su func. Añadir im.

\subsection{}{Esquema de conexionado}

Explicación breve (tabla?). Ref al anexo.


\section{Diseño Software}\label{sec:dissoft}

\subsection{Código Arduino}

\subsubsection{Conexión Wifi y MQTT}

\subsubsection{Estructura de datos}

\subsubsection{Obtención de datos de sensores}

\subsubsection{Manejo de errores en los sensores}

\subsubsection{Obtención de hora}

\subsubsection{Serializado de datos}

\subsubsection{Envío de datos}

\subsubsection{Actualización OTA}

\subsubsection{Pantalla OLED}

\subsubsection{Deep-sleep}

\subsubsection{Flujo global del programa}

\subsection{Node-RED}

\subsubsection{Muestra de datos recibidos}

\subsubsection{Almacenamiento de datos en MongoDB}

\subsubsection{Acceso de datos en MongoDB}

\subsubsection{Envío de configuración al dispositivo}
\section{Estudio de consumo y optimización de energía}\label{sec:energia}
\section{Resultados y conclusiones}\label{sec:conclusion}

\subsection{Resultados}

¿Qué hemos obtenido literalmente? Estudios de consumo.

\subsection{Conclusiones}

Crítica sobre lo bien que hemos trabajado en IoT y en grupo, organización, buenos resultados, flexibilidad...

Crítica a qué se podría mejorar de cada parte. Trabajo futuro por cada parte.
\printbibliography
\section{Manual de usuario}\label{sec:manual}

Como se distribuye la interfaz de usuario. Como configurar el dispositivo. Que cosas hay automáticas y que manuales. 
\section{Lista de ficheros}\label{sec:listaficheros}

Lista de archivos sin más.

\end{document}
